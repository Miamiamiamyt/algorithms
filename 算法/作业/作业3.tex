\documentclass[19pt,a4paper]{article}
\usepackage{xeCJK}
\usepackage{amsmath}
\setmainfont{STSong}
\usepackage{geometry}
\geometry{left=2.5cm,right=2.5cm,top=2.5cm,bottom=2.5cm}
\setlength{\parindent}{4em}
\title{算法作业3}
\author{孟妍廷 2015202009}
\date{2017年10月7日}

\begin{document}
\maketitle

5.2-5\\
\indent 解:\\
\indent \ \ 首先,假设$X_{ij}$对应(i,j)对为A的逆序对该事件的指示器随机变量:
$$X_{ij}=I\{(i,j)为逆序对\}=
\begin{cases}
1,& \text{i<j,A[i]>A[j]}\\
0,& \text{i<j,A[i]<A[j]}
\end{cases}$$
\indent 由于数列A中的n个数均不相同,所以对于数对(i,j)(i<j),只有$A[i]<A[j],A[i]>A[j]$两种可能,\\
\indent 且A的元素是1-n的均匀随机排列,故Pr{(i,j)}=$\frac{1}{2}$.\\
\indent 根据定理,E[$X_{ij}$]=Pr\{(i,j)为A的逆序对\}=$\frac{1}{2}$\\
\indent 设X为一个随机变量,其值等于数列A中逆序对的个数,故X可表示为:\\
\begin{equation*}
X = \sum_{i=1}^{n-1}\sum_{j=i+1}^nX_{ij}
\end{equation*}
\indent 则逆序对的数目期望为
$$E[X] = E[ \sum_{i=1}^{n-1}\sum_{j=i+1}^nX_{ij}] $$
$$= \sum_{i=1}^{n-1}\sum_{j=i+1}^nE[X_{ij}] $$
$$=\sum_{i=1}^{n-1}\sum_{j=i+1}^n\frac{1}{2}$$
$$=\frac{(n-1+1)\times(n-1)}{2}\cdot\frac{1}{2}$$
$$=\frac{n(n-1)}{4}$$
\\
\\
\indent 5.3-5\\
\indent 证明:\\
\indent\ \ 从1-$n^3$中随机取n个数,允许重复的情况下总共有${n^3}^n$种可能\\
\indent 不允许重复的情况下随机取n个数,总共有
$$P=(n^3)\times(n^3-1)\times\cdot\cdot\cdot\times(n^3-n+1)$$
\indent 种可能.\\
\indent 故所有元素都唯一的概率为:\\
$$Pr=\frac{P}{{n^3}^n}$$
$$=\frac{(n^3)\times(n^3-1)\times\cdot\cdot\cdot\times(n^3-n+1)}{{n^3}^n}$$
\indent 接下来对概率进行放缩:
$$Pr>\frac{{(n^3-n)}^n}{{n^3}^n}$$
$$={\frac{(n^3-n)}{n^3}}^n$$
$$={(1-\frac{1}{n^2})}^n\ \ \ \ \ \ (*)$$
\indent 对多项式(*)进行展开:
$$(*)=1+n\times(-\frac{1}{n^2})+\cdot\cdot\cdot>1-\frac{1}{n}$$
\indent 得证.
\\
\\
\indent 5.4-2\\
\indent 解:\\
\indent\ \ 假设$n_i$为投球次数,则依题意可得
$$P(n_i=1)=0$$
$$P(n_i=2)=1\times\frac{1}{b}=\frac{1}{b}$$
$$P(n_i=3)=1\times\frac{b-1}{b}\times\frac{2}{b}=\frac{2(b-1)}{b^2}$$
$$\cdot$$
$$\cdot$$
$$\cdot$$
$$P(n_i=k)=1\times\frac{b-1}{b}\cdot\cdot\cdot\times\frac{b-k+2}{b}\times\frac{k-1}{b}=\frac{(k-1)b(b-1)\cdot\cdot\cdot(b-k+2)}{b^{k}}$$
$$\cdot$$
$$\cdot$$
$$\cdot$$
\\
\indent 故投球次数期望为:
$$E\sum_{i=1}^{i=b+1}i\times\frac{(i-1)b(b-1)\cdot\cdot\cdot(b-i+2)}{b^{i}}$$



\end{document}