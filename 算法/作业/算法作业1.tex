\documentclass[19pt]{article}
\usepackage{xeCJK}
\usepackage{amsmath}
\setmainfont{STSong}
 \usepackage{geometry} 
\geometry{left=2.5cm,right=2.5cm,top=2.5cm,bottom=2.5cm}
\usepackage{indentfirst}
\setlength{\parindent}{4em}
\title{算法作业1}
\author{孟妍廷 2015202009}
\date{}

\begin{document}
\maketitle


3.1.7\\
\indent 证明:\quad 假设$f_1(x)=o(g(n))$,\ $f_2(x)=\omega(g(n))$,则由定义可得:\\
\indent 对任意正常量$c>0$,\ 存在常量$n_1>0$,\ 使得对所有$n\ge n_0$, 有$0 \le f_1(n)<cg(n)$;同理,对于同一常量c,\ 存在常量$n_1>0$,\ 使得对所有$n\ge n_1$,\ 有$f_2(n)>cg(n)$. \\
\indent 取$N=max(n_0,n_1)$,\ 则对所有$n\ge N$,\ 有$f_1(n)<cg(n)$且$f_2(n)>cg(n)$,\ 即$o(n)<cg(n)$且 $\omega(n)>cg(n)$. \\
\indent 故$o(n)\cap \omega(n)=\emptyset$得证。\\

3-3\\
\indent a. 解:按照渐进增长率和界限函数可得自上而下排序为(同一行的为等价类):\\
\indent $2^{2^{n+1}}$ \quad $2^{2^n}$\\
\indent $(n+1)!$\\
\indent $n!$\\
\indent $e^n$\\
\indent $n\cdot2^n$\\
\indent $2^n$\\
\indent $(\frac{3}{2})^n$\\
\indent $n^{\lg \lg n}$\quad $(\lg n)^{\lg n}$\\
\indent $(\lg n)!$\\
\indent $n^3$\\
\indent $n^2$\quad$4^{\lg n}$\\
\indent $n\lg n$\quad $\lg (n!)$ 由夹逼定理$n^n$与$n!$同阶\\
\indent $n$\quad$2^{\lg n}$\\
\indent $(\sqrt{2}^{\lg n})$\\
\indent $2^{\sqrt{2\lg n}}$\\
\indent $\lg ^2n$\\
\indent $\ln n$\\
\indent $\sqrt {\lg n}$\\
\indent $\ln \ln n$\\
\indent $2^{lg^*n}$\\
\indent $lg^*n$\\
\indent $\lg ^*(\lg n)$\\
\indent $\lg (\lg ^*n)$\\
\indent $n^{\frac{1}{\lg n}}$\quad$1$\\
\\
\indent b.解:\\
\indent 依题意可得:当该非负函数$f(n)$的极限不存在时,不存在$g(n)$,使$f(n)$是$O(g(n))$或$\Omega(g(n))$的.\\
\indent 故f(n)的一个例子是:\\
\indent $$f(x)=\begin{cases}0,&\text{x=2k+1}\\1,&\text{x=2k}\end{cases}$$\\
\indent 其中, $k\in Z$.\\
\\
\indent3-4\\
\indent a.错误。反例:\ $n=O(n^2)$但是$n^2\ne O(n)$\\
\indent b.错误。反例:\ $n^2+n=\Theta(n^2)\ne \Theta(n)$\\
\indent c.正确。 证明:\ 由$f(n)=O(g(n))$表明对任意正常数c,存在常量$n_0$,对所有$n\ge n_0$都有$f(n)\le cg(n)$,\\
所以也有$\lg f(n)\le\lg (cg(n))$,故$\lg (f(n))=O(\lg (g(n))) $成立.\\
\indent d.正确。 证明:\ 由$f(n)=O(g(n))$表明对任意正常数c,存在常量$n_0$,对所有$n\ge n_0$都有$f(n)\le cg(n)$,\\
由于f(n)和g(n)是正函数,所以也有$2^{f(n)}\le2^{cg(n)}$,故$2^{f(n)}=O(2^{g(n)}) $成立.\\
\indent e.错误。 反例:\ $f(n)=\frac{1}{n}$,则$f(n)^2=\frac{1}{n^2}$,故$\frac{1}{n}\ne O(\frac{1}{n^2})$\\
\indent f.正确。 证明:\ 由$f(n)=O(g(n))$可得$${ \lim_{n \to +\infty} \frac{f(n)}{g(n)} \le c_1}$$因此
$${ \lim_{n \to +\infty} \frac{g(n)}{f(n)} \ge c_2}$$
\indent 其中$c_1$,$c_2$为常数,故$g(n)=\Omega(f(n))$。\\
\indent g.错误。 反例:\ $n^n\ne \frac{n}{2}^{\frac{n}{2}}$\\
\indent h.错误。 反例:\ $n+n^2\ne \Theta(n)$



\end{document}