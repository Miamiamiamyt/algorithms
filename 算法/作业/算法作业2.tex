\documentclass[19pt,a4paper]{article}
\usepackage{xeCJK}
\usepackage{amsmath}
\setmainfont{STSong}
 \usepackage{geometry} 
\geometry{left=2.5cm,right=2.5cm,top=2.5cm,bottom=2.5cm}
\setlength{\parindent}{4em}
\title{算法作业2}
\author{孟妍廷 2015202009}
\date{2017年9月24日}
\begin{document}
\maketitle

1.给出递归式主方法的推导过程\\
\indent \quad 解:假设已知$a\ge 1,b>1$为常数,$f(n)$是一个函数,$T(n)$由以下递归式定义:$$T(n)=aT(\frac{n}{b})+f(n).$$
\indent 则利用迭代法可得:\\
$$T(n)=aT(\frac{n}{b})+f(n)$$
$$=a^2T(\frac{n}{b^2})+af(\frac{n}{b})+f(n)$$
$$=a^3T(\frac{n}{b^3})+a^2f(\frac{n}{b^2})+af(\frac{n}{b})+f(n)$$
$$=\cdot\cdot\cdot$$
\indent 迭代$l$次时:
$$=f(n)+af(\frac{n}{b})+\cdot\cdot\cdot+a^{l-1}f(\frac{n}{b^{l-1}})+a^lT(\frac{n}{b^l})$$
$$=\sum_{i=0}^{l-1}a^if(\frac{n}{b^i})+a^lT(\frac{n}{b^l})\ \ \ \ \ \ \ *$$
\indent 其中,$\frac{n}{b^l}\le 1$,即$l\ge log_bn$,故:
$$(*)\le \sum_{i=0}^{l-1}a^if(\frac{n}{b^i})+a^{ log_bn}T(1)$$
$$\le \sum_{i=0}^{l-1}a^if(\frac{n}{b^i})+c\cdot a^{ log_bn}$$
$$= \sum_{i=0}^{l-1}a^if(\frac{n}{b^i})+c\cdot  n^{log_ba}\ \ \ \ \ \ \ **$$
\indent 故对于任意$\varepsilon > 0$:\\
\indent (1)若有$f(n)=O(n^{log_ba-\varepsilon})$,则
$$(**)=O(\sum_{i=0}^{log_bn-1}a^i{(\frac{n}{b^i})}^{log_ba-\varepsilon})+c\cdot  n^{log_ba}$$
$$=O(\sum_{i=0}^{log_bn-1}\frac{a^i}{b^{i(log_ba-\varepsilon)}}n^{log_ba-\varepsilon})+c\cdot n^{log_ba}$$
$$=O(\sum_{i=0}^{log_bn-1}b^{i\varepsilon}n^{log_ba-\varepsilon})+c\cdot n^{log_ba}$$
$$=O(\frac{n^\varepsilon-1}{b^\varepsilon-1}n^{log_ba-\varepsilon})+c\cdot n^{log_ba}$$
$$=O(n^{log_ba})+c\cdot n^{log_ba}$$
\indent 故$T(n)=\Theta(n^{log_ba})$.\\

\indent (2)若有$f(n)=\Theta(n^{log_ba})$,则
$$(**)=\Theta(\sum_{i=0}^{log_bn-1}a^i{(\frac{n}{b^i})}^{log_ba})+c\cdot  n^{log_ba}$$
$$=\Theta(\sum_{i=0}^{log_bn-1}\frac{a^i}{b^{ilog_ba}}n^{log_ba})+c\cdot n^{log_ba}$$
$$=\Theta(\sum_{i=0}^{log_bn-1}\frac{a^i}{a^i}n^{log_ba})+c\cdot n^{log_ba}$$
$$=\Theta(log_bn\cdot n^{log_ba})+c\cdot n^{log_ba}$$
$$=\Theta(\frac{1}{logb}\cdot n^{log_ba}logn)+c\cdot n^{log_ba}$$
\indent 由于$\frac{1}{logb}$为常数,且$n^{log_ba}logn$的渐进增长率大于$ n^{log_ba}$\\
\indent 故$T(n)=\Theta(n^{log_ba}logn)$.\\

\indent (3)若有$f(n)=\Omega(n^{log_ba+\varepsilon})$,且对某常数$c_1>1$和足够大的n,有$af(\frac{n}{b})\le cf(n)$,则
$$(**)\le \sum_{i=0}^{log_bn-1}c_1^if(n)+c\cdot  n^{log_ba}$$
$$=f(n)\cdot \frac{1-c_1^{log_bn}}{1-c_1}+c\cdot  n^{log_ba}$$
\indent 由于$\frac{1-c_1^{log_bn}}{1-c_1}$为常数,且$f(n)$的渐进增长率大于$ n^{log_ba}$\\
\indent 故$T(n)=\Theta(f(n))$.\\
\\
\indent 主定理推理完毕.


\end{document}