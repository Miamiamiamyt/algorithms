\documentclass[19pt,a4paper]{article}
\usepackage{xeCJK}
\usepackage{amsmath}
\setmainfont{STSong}
\usepackage{geometry}
\geometry{left=2.5cm,right=2.5cm,top=2.5cm,bottom=2.5cm}
\setlength{\parindent}{4em}
\usepackage{graphicx}
\title{思考题7}
\author{孟妍廷2015202009}
\date{2017年11月7日}

\begin{document}
\maketitle
\indent 依题意可得,最优直方图问题符合最优子结构性质,适合用动态规划解决问题.\\
\indent 设$SSE^*[i,k]$表示将排序后的数组中第1个到第i个元素放入k个桶中的最小二次求和误差.\\
\indent AVG[i,j]表示排序后的第i到第j个元素在一个桶中的均值.\\
\indent 算法如下\\
\indent 首先对原始数组a[1...n]由小到大排序,得到order[1...n]记录下标.\\
\indent $COUNT(order,SSE*,AVG,PP,i,k)\\
\indent \quad if\ k==1\\
\indent \quad\quad cost=SSE*[j,1]\\
\indent \quad\quad for\ j=1\ to\ i-1\\
\indent \quad\quad\quad cost=\infty\\
\indent \quad\quad\quad temp=COUNT(order,SSE*,AVG,PP,j,k-1)+PP[i]-PP[j+1]-(i-j)\times AVG[j,i]^2 \\
\indent \quad\quad\quad (cost>temp)?cost=temp:cost=cost\\
\indent \quad return\ cost\\$
\\
\indent $SOLUTION(order,n)\\
\indent \quad let\ AVG[1...n,1...n],SSE^*[1...n,1...B],P[0...n],PP[0...n]\ be\ new\ array.\\
\indent \quad P[0]=0,PP[0]=0\\
\indent \quad for\ i=1\ to\ n\\
\indent \quad\quad for\ j=1\ to\ i\\
\indent \quad\quad\quad P[i]+=a[order[i]]\\
\indent \quad\quad\quad PP[i]+=a[order[i]]^2\\
\indent \quad for\ i=1\ to\ n\\
\indent \quad\quad for\ j=i\ to\ n\\
\indent \quad\quad\quad AVG[i,j]=\frac{P[j]-P[i-1]}{j-i+1}//计算出所有平均数\\
\indent \quad\quad\quad SSE^*[i,j]=\infty\\
\indent \quad for\ i=1\ to\ n\\
\indent \quad\quad SSE^*[i,1]=PP[i]-i\times AVG[1,i]^2//所有数放入一个桶中的情况的误差可以直接算\\
\indent \quad for\ k=1\ to\ B\\
\indent\quad\quad res=\infty\\
\indent \quad\quad if\ res>COUNT(order,SSE*,AVG,PP,n,k)\\
\indent \quad\quad \quad res=COUNT(order,SSE*,AVG,PP,n,k)\\
\indent \quad\quad \quad key=k\\
\indent \quad return\ key\ res\\$
\\
\indent 时间复杂度为$O(B\times B\times n)$

\end{document}
