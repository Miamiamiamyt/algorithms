\documentclass[19pt,a4paper]{article}
\usepackage{xeCJK}
\usepackage{amsmath}
\setmainfont{STSong}
\usepackage{geometry}
\geometry{left=2.5cm,right=2.5cm,top=2.5cm,bottom=2.5cm}
\setlength{\parindent}{4em}
\title{思考题3}
\author{孟妍廷 2015202009}
\date{2017年10月10日}

\begin{document}
\maketitle
这次思考题感觉比较难理解,一开始看了特征序列结果发现不是很适用,尝试着做一下......\\
\\
\\
\indent\ \ 1.解:\\
\indent 设x为最大的盒子中的小球数,已知有b个小球b个盒子,所以x最大为b,最小为1\\
$$Pr(x=b)=\frac{b}{b^b}<\frac{1}{(b-1)!}$$
$$Pr(x=b-1)=\frac{b(b-1)}{b^b}<\frac{2}{(b-2)!}$$
$$\cdot$$
$$\cdot$$
$$\cdot$$
$$Pr(x=b-i)=\frac{b\cdot\cdot\cdot(b-i)}{b^b}<\frac{i+1}{(b-i-1)!}$$
\indent 故最大的盒子中的小球数的期望为:\\
$$E(x)=\sum_{i=1}^{b}(b-i)\times\frac{i+1}{(b-i-1)!}$$
$$<\sum_{i=1}^{b}\frac{(i+1)^2}{i!}$$
$$<logb的Taylor展开$$
$$<logb=O(logb)$$
\\
\\
\indent\ \ 2.解:\\
\indent 设x为最大的盒子中的小球数,已知有b个小球b/logb个盒子,所以x最大为b,最小为logb\\
\indent 我认为第一种方法与1.同理。\\
\indent 第二种方法思路:\\
\indent 算出x的取值>clogb的概率小于某个值,其中c为一个常数\\
\indent 由于\\
$$E(x)=\sum_{i=logb}^{b}i\times Pr(x=i)$$
$$=\sum_{logb}^{clogb}i\times Pr(x=i)+\sum_{clogb+1}^{b}i\times Pr(x=i)$$
$$< c_1logb+一个很小的数(c_1为常数)$$
$$=O(logb)$$
\\
\\
\indent\ \ 3.第三题。。。不太会做。。。

\end{document}
