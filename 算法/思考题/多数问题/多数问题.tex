\documentclass[19pt,a4paper]{article}
\usepackage{xeCJK}
\usepackage{amsmath}
\setmainfont{STSong}
\usepackage{geometry}
\geometry{left=2.5cm,right=2.5cm,top=2.5cm,bottom=2.5cm}
\setlength{\parindent}{4em}
\title{思考题1}
\author{孟妍廷 2015202009}
\date{2017年9月27日}

\begin{document}
\maketitle
\indent 分析:对于多数问题,给定一个大小为n的数组,则n只有为偶数和为基数两种可能,分情况讨论:\\
\indent 1.偶数:当n为偶数时,若一个数出现的次数要超过$\frac{n}{2}$,该数起码有两个是相邻的,示例如下:\\
$$121314\ 不满足$$
$$113141\ \ \ 满足$$
\indent 2.奇数:当n为奇数时,若一个数出现的次数要超过$\frac{n}{2}$,该数在数组中出现的最离散的情况如下:\\
$$1213141\ 每两个1之间有一个不为1的数间隔$$
\indent 根据以上特性,可以设置一个变量count(初值为1),假定当前数组存在一个数出现次数超过$\frac{n}{2}$,\\
\indent 首先假设数组第一位的数就是这个数,向后遍历,当这个数再次出现时count++,当这个数没有出\\
\indent 现时count--,当count减到零时,假定数组下一位是这个数.\\
\indent 若到最后一位前都未找到,则假定数组最后一位就是这个数,进行检查.\\
\indent 该算法的时间复杂度为O(n).伪码如下:\\
\indent $int\ majority(int\ num[\ ],int\ begin,int\ end) \ \{ $\\
\indent $\ \ if(begin\ ==\ end\ \&\&\ check(end)\ ==\ 1)$\\
\indent $\ \ \ \ return\ end;$\\
\indent $ \ \ int\ count\ =\ 1; $\\
\indent $ \ \ int\ i \ =\ begin; $\\
\indent $ \ \ while(count\ >\ 0\ \&\&\ i\ <\ end)\ \{$\\
\indent $ \ \ \ \ if(num[++i]\ ==\ num[begin])$\\
\indent $ \ \ \ \ \ \ count++;$\\
\indent $ \ \ \ \ else$\\
\indent $ \ \ \ \ \ \ count--;$\\
\indent $\ \ \}$\\
\indent $\ \ if(count\ >\ 0)$\\
\indent $\ \ \ \ return\ begin;$\\
\indent $\ \ else\ $\\
\indent $\ \ \ \ majority(num[n],++begin,end);$\\
\indent $\}$\\
\indent 因此,首先以begin为0,end为n开始调用majority函数即可,空间复杂度为O(1).

\end{document}